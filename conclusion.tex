\section{Conclusion}
In this paper, we study parallel algorithms for bi-core decomposition, which is an important theoretical problem with many real-world applications. We develop the first shared-memory work-efficient parallel bi-core decomposition algorithm with nontrivial span bounds. Our parallel algorithm improves the span complexity from the state-of-the-art $\BigO{m}$ to $\BigO{\rho \log(n)}$ \textit{w.h.p.}. Practically, the span we achieve is 2--3 orders of magnitude lower than the span of existing parallel algorithm. Further, we prove the problem of bi-core decomposition to be P-complete. We also introduce a parallel indexing structure to store the bi-cores, and provide a work-efficient parallel index construction algorithm and query algorithm. Finally, we introduce optimizations such as peeling space pruning and provide optimized implementations of our algorithms. We perform experimental evaluation of our algorithms on real-world bipartite graphs. Our parallel algorithms outperform sequential baselines by up to 44x for peeling and 22.3x for query. Experimental results also prove our bi-core decomposition algorithms are capable of scaling to real-world graphs with hundreds of millions of edges, processing them in minutes and performing bi-core queries in milliseconds. 

% In the future, we hope to further optimize the actual implementation of our algorithm and obtain better runtime results. We also intend to provide a complexity proof to show that the problem of bi-core decomposition is P-complete. We aim to bound the bi-core peeling complexity constant in our span. We also plan to parallelize the index construction and maintenance algorithms introduced Liu \textit{et al.} \cite{Liu2020Efficient}. Moreover, we will perform more comprehensive experiments to demonstrate and analyze the performance of our algorithms. As extension to the work on bi-core, we may also investigate parallel bi-clique peeling or parallel community search in bipartite graphs. 